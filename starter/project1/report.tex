\documentclass[11pt]{article}
\usepackage{amsmath}
\usepackage{amsfonts}
\usepackage{amsthm}
\usepackage[utf8]{inputenc}
\usepackage[margin=0.75in]{geometry}

\title{CSC111 Winter 2025 Project 1}
\author{ECE AVSAR, MOHAMED ABDELFATTAH}
\date{\today}

\begin{document}
\maketitle

\section*{Running the game}
We should be able to run your game by simply running \texttt{adventure.py}.  
If you have any other requirements (e.g., installing certain modules), describe them here.  
Otherwise, skip this section.

\section*{Game Map}
Example game map below (edit it to show your actual game map):

\begin{verbatim}
  -1 -1 1  2 3
  -1  7 4  5 6
  -1 -1 -1 8 -1
\end{verbatim}

Starting location is: 1

\section*{Game solution}
List of commands:
\begin{verbatim}
["take toonie", "go east", "use toonie", "take key", "go east", "use key",
 "go south", "take USB driver", "go west", "1", "go west", "2550",
 "take laptop charger", "take UofT mug", "go east", "go east", "go south",
 "use USB driver", "use laptop charger", "use UofT mug"]
\end{verbatim}

\section*{Lose condition(s)}
Description of how to lose the game:  
If the player moves more than 30 times within the map without bringing the required items  
to their dorm AND using them, they lose the game.

List of commands: 
\begin{verbatim}
["go east", "inventory", "use toonie", "take key", "inventory", "go east",
 "use key", "look", "go south", "take USB driver", "go west", "score", "7",
 "6", "1", "go west", "4350", "2550", "take laptop charger", "take UofT mug",
 "go east", "go east", "go east", "undo", "go south", "use Usb driver",
 "score", "use laptop charger", "use UofT mug"]
\end{verbatim}

\textbf{Code parts involved:}  
In \texttt{adventure.py}, from line 291-321 (to check if the player takes the item or uses it  
at correct positions and affects the inventory accordingly) and 345-347 (for game over screen).  
If the player does not take and use the items in their starting and target locations,  
and if the player's number of commands reaches or exceeds 30, they lose the game.

\section*{Inventory}

\begin{enumerate}
    \item \textbf{All location IDs that involve items in the game:}

    \item \textbf{Item data:}
    \begin{enumerate}
        \item \textbf{Item 1:}
        \begin{itemize}
        \item Name: \texttt{"toonie"}
        \item Start location ID: 1
        \item Target location ID: 2
        \end{itemize}
        \item \textbf{Item 2:}
        \begin{itemize}
        \item Name: \texttt{"key"}
        \item Start location ID: 2
        \item Target location ID: 3
        \end{itemize}
        \item \textbf{Item 3:}
        \begin{itemize}
        \item Name: \texttt{"ancient book"}
        \item Start location ID: 3
        \item Target location ID: 3
        \end{itemize}
    \end{enumerate}

    \item \textbf{Exact command(s) for picking up and using/dropping items:}  
    \texttt{"take [item]"} and \texttt{"use [item]"}

    \item \textbf{Code parts involved:}  
    In \texttt{adventure.py}, \texttt{AdventureGame} class (inventory attribute),  
    \texttt{get\_inventory\_items} function from lines 231-235.
\end{enumerate}

\section*{Score}
\begin{enumerate}
    \item \textbf{How players earn scores:}  
    Players earn scores by taking items from their starting position and using them in their target position.  
    Example:  
    \begin{verbatim}
    ["take toonie", "go east", "use toonie"]
    \end{verbatim}

    \item \textbf{Score tracking demo:}
    \begin{verbatim}
    ["take toonie", "go east", "use toonie", "score"]
    \end{verbatim}

    \item \textbf{Code parts involved:}  
    In \texttt{adventure.py}, lines 225-229.
\end{enumerate}

\section*{Enhancements}
\begin{enumerate}
    \item \textbf{NPC Class}
    \begin{itemize}
        \item Description: Implemented an NPC class that increases player engagement  
        with attributes such as messages and interactions to solve puzzles.
        \item Complexity level: High
        \item Reason: Required modifying \texttt{game\_entities.py}, adding NPCs to JSON,  
        and adjusting \texttt{\_load\_game\_data}.
    \end{itemize}

    \item \textbf{Location Class Enhancements}
    \begin{itemize}
        \item Description: Added attributes \texttt{door}, \texttt{password}, and \texttt{is\_unlocked}  
        to track locked doors and implement a password puzzle.
        \item Complexity level: Medium
        \item Reason: Adjustments were needed in \texttt{Location} class, but it wasn’t overly complex.
    \end{itemize}

    \item \textbf{Key-Lock Puzzle and Math Problems}
    \begin{itemize}
        \item Description: Added puzzles involving retrieving a key, solving math problems, and unlocking doors.
        \item Complexity level: Low
        \item Reason: Involved minor code adjustments, mainly in \texttt{adventure.py}  
        (lines 314-316, 329-334, 336-343).
    \end{itemize}

    \item \textbf{Unlock Function}
    \begin{itemize}
        \item Description: Implemented an \texttt{unlock} command.
        \item Complexity level: Medium
        \item Reason: Required three attributes in \texttt{Location} class and code changes in  
        \texttt{adventure.py} (lines 277-289).
    \end{itemize}
\end{enumerate}

\end{document}
